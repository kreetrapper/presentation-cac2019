\RequirePackage{snapshot}
\documentclass[]{beamer}

% don't display the beamer navbar
\usenavigationsymbolstemplate{}

\usepackage[german,english]{babel}
% \usepackage{yfonts}         % old German fonts
% \usepackage[safe]{tipa}     % IPA fonts

% \usepackage{tikz}           % create graphics programmatically
% \usetikzlibrary{arrows}
% \tikzstyle{block}=[draw opacity=0.7,line width=1.4cm]

%%%%%
%%% BIBLIOGRAPHY
%
\usepackage[
%    style=authoryear,  % currently, only works 
    natbib=true,
    defernumbers=true
]{biblatex}
\addbibresource{bibliography.bib}
% to ensure that quotes are set according to the rules of the main languae
\usepackage{csquotes}

\DeclareBibliographyCategory{online}
\DeclareBibliographyCategory{article}
\DeclareBibliographyCategory{book}
\DeclareBibliographyCategory{triangle}

\AtDataInput{%
    \ifboolexpr{%
        test {\ifboolexpr{%
            test {\ifentrytype{misc}} 
            or 
            test {\ifentrytype{online}}}}
        and
        not test {\iffieldundef{url}}
    }{%
        \addtocategory{online}{\thefield{entrykey}}
    }{%
        \ifboolexpr{%
            test {\ifentrytype{article}} 
            or 
            test {\ifentrytype{inproceedings}}
            or
            test {\ifentrytype{incollection}}
            or
            test {\ifentrytype{thesis}}
        }{%
            \addtocategory{article}{\thefield{entrykey}}
        }{%
            \ifboolexpr{%
                test {\ifentrytype{book}}
                or
                test {\ifentrytype{collection}}
            }{%
                \addtocategory{book}{\thefield{entrykey}}
            }{%
                \addtocategory{triangle}{\thefield{entrykey}}
            }
        }
    }
}
%
%%%
%%%%%

\usepackage{xmpmulti}
% \usepackage{multimedia}     % used for animations

% Produce 'blind' text for testing.
\usepackage{blindtext}



\usepackage{graphicx}

\usetheme{Eurac}

\newcommand{\myauthor}{Alexander König, Egon W. Stemle}
\newcommand{\myshortauthor}{König, Stemle}
\newcommand{\mymail}{Firstname.Lastname@eurac.edu}
\newcommand{\mytitle}{Technical Solutions for Reproducible Research}
%\newcommand{\mytitle}{My Long and Exciting Title -- Pretty Damn Long! Wouldn't You Say? ..... ........ ...... .. ......... ......... .....}
%\newcommand{\mytitle}{My Long and Exciting Title -- Pretty Damn Long! Wouldn't You Say? \ldots and it's not over quite, yet!}
\newcommand{\myshorttitle}{CAC 2019: Reproducibility}
%\newcommand{\mysubtitle}{very short subtitle}
\newcommand{\mysubtitle}{CLARIN Annual Conference 2019}
%\newcommand{\mysubtitle}{not-so-interesting but-still-important-subtitle that-hasn't-ended-just-yet}
\newcommand{\mypdftitle}{\mytitle -- \mysubtitle}
\newcommand{\myinstitute}{Institute for Applied Linguistics \\ Eurac Research, Bozen/Bolzano, Italy}
\newcommand{\myshortinstitute}{ComMul}
\newcommand{\mydate}{2 October 2019}
\newcommand{\mykeywords}{reproducibility, docker, git}

\title[\myshorttitle]{\mytitle}
\subtitle{\mysubtitle}
\author[\myshortauthor]{\myauthor \\ {\tt<\mymail>}}
\institute[\myshortinstitute]{\myinstitute}
\keywords{\mykeywords}
\subject{Slides of presentation:\myshorttitle, from \mydate.}
\date{\mydate}

%
%DOCUMENT begins here
%
\begin{document}

\begin{frame}[plain,noframenumbering]
	\titlepage
\end{frame}

%
% OVERVIEW
%
%\section*{}
%    \setbeamertemplate{footline}{}
%    \begin{frame}<handout:0>[noframenumbering]
%        \frametitle{Outline}
%        \tableofcontents[currentsection]
%    \end{frame}
%    \begin{frame}<beamer:0>[noframenumbering]
%        \frametitle{Outline}
%        \tableofcontents[]
%    \end{frame}


%%%%%%
%
%CONTENT comes here
%

%
% Corpus from the WEB
%
%\section{Corpora}
\section[p.one]{Introducing the Problem}
{%
\begin{frame}<beamer>
    \frametitle{Outline}
\tableofcontents[currentsection]
\end{frame}
}

	\begin{frame}
		\frametitle{Introduction}
		\begin{block}{The Problem}
			\begin{itemize}
				\item By design, research results should be verifiable by other researchers to ensure good scientific practice and easily detect mistakes.
				\pause
				\item While this is a key feature of all research, it seems that this can often be a problem - especially in the Social Sciences and Humanities (SSH).
				\pause
				\begin{itemize}
					\item Primary data is often not available to fellow researchers.
					\pause
					\item The analysis is based on software that is obscure, not easily available or (at worst) not properly identified in the scientific publication.
				\end{itemize}
		   \end{itemize}
	   \end{block}
	\end{frame}

	\begin{frame}
		\frametitle{Introduction}
		\begin{block}{The Question}
			\begin{itemize}
				\pause
				\bigskip
				\item \textbf{How can we make Linguistic Research more reproducible?}
				\bigskip
		   \end{itemize}
	   \end{block}
	\end{frame}

    \begin{frame}
        \frametitle{Introduction}
        \begin{block}{Data Sharing}
            \begin{itemize}
				\item Both the primary data and the tools being used have to be made available to fellow researchers.
				\pause
				\item If the data and tools can be shared publicly this has the added bonus that they can be re-used for other research as well.
				\pause
				\item Good guidelines for how research data should be shared can be found in the FAIR Principles (\url{https://www.go-fair.org/fair-principles/}).
	     	\end{itemize}
        \end{block}
    \end{frame}


	\begin{frame}
        \frametitle{Introduction}
        \begin{block}{Versioning of Data and Tools}
            \begin{itemize}
				\item Linguistic corpora are often "living data", which means they constantly keep being improved and added onto.
				\pause
				\item All versions of a corpus that have been the basis of a scientific analysis have to be available.
				\pause
				\item The same is true for linguistic tools that are being used to process the data.
	     	\end{itemize}
        \end{block}
    \end{frame}

\section[p.two]{Possible Solutions}
{%
\begin{frame}<beamer>
    \frametitle{Outline}
\tableofcontents[currentsection]
\end{frame}
}

    \begin{frame}
        \frametitle{Reproducible Data}
        \begin{block}{Versioning of linguistic corpora}
            \begin{itemize}
				\item Most linguistic corpora are text-based or have a text component (and it's especially this component that is changing).
				\pause
				\item An existing versioning software like subversion or git can be used to track changes in the primary data.
				\pause
				\item To make the various versions available and have the changes be transparent, the data can be hosted on a Code Hosting Site like github or gitlab.
	     	\end{itemize}
        \end{block}
    \end{frame}

    \begin{frame}
        \frametitle{Reproducible Methods}
        \begin{block}{Versioning of linguistic tools}
            \begin{itemize}
				\item In linguistic research handcrafted toolchains built out of a variety of separate programs are very common.
				\pause
				\item Often, it will be difficult to rebuild such a toolchain exactly.
				\pause
				\begin{itemize}
					\item Some tools might no longer be available or cannot be found.
					\pause
					\item It might not be completely clear which specific version of a tool was used.
					\pause
					\item Some manufacturers do not keep older versions of their software available for download.
				\end{itemize}
				\pause
				\item One solution is to create a (Docker) container with a "frozen" version of the complete toolchain.
				\pause
				\item Such a container can also be made available in a public container registry.
				\pause
				\item Orchestrators such as Kubernetes can help fellow researchers to easily deploy such a container to reproduce the analysis.
	     	\end{itemize}
        \end{block}
    \end{frame}

\section[p.three]{A Case Study}
{%
\begin{frame}<beamer>
    \frametitle{Outline}
\tableofcontents[currentsection]
\end{frame}
}

    \begin{frame}
        \frametitle{Case Study}
        \begin{block}{The MERLIN corpus}
            \begin{itemize}
				\item At the Eurac Research CLARIN Centre (ERCC) we made some first steps in implementing these ideas.
				\pause
				\item By now multiple corpora are versioned via git, the multilingual MERLIN corpus being the first.
				\pause
				\item The whole corpus is available on an on-premise gitlab installation.
				\pause
				\item The different versions of the corpus are realized as git tags.
				\pause
				\item Tagged versions are also uploaded into a CLARIN DSpace repository.
				\pause
				\item The DSpace and the gitlab repository are pointing at each other, so users can choose their preferred way of obtaining the data.
				\pause
				\item {\small \url{https://gitlab.inf.unibz.it/commul/merlin-platform/data-bundle}}
				\pause
				\item \url{http://hdl.handle.net/20.500.12124/6}
     		\end{itemize}
        \end{block}
    \end{frame}

\section[p.four]{Challenges and Pitfalls}
{%
\begin{frame}<beamer>
    \frametitle{Outline}
\tableofcontents[currentsection]
\end{frame}
}

    \begin{frame}
        \frametitle{Challenges and Pitfalls}
        \begin{block}{Some examples of problems - encountered and expected}
            \begin{itemize}
				\pause
				\item[\textbf{(?)}] How to handle non-public data? Hide the git repository? Protect it with a password?
				\pause
				\item[\textbf{(!)}] We used password-protected git submodules to keep the general description (README,CHANGELOG) still accessible to everybody.
				\pause
				\item[\textbf{(?)}] Can non-local infrastructure like github be trusted with sensitive data?
				\pause
				\item[\textbf{(!)}] gitlab is available as open source and can be installed locally, meaning the data will never leave the researcher's control.
				\pause
	 			\item[\textbf{(?)}] Dockerfiles do not enforce consistent versioning of installed packages.
				\pause
				\item[\textbf{(!)}] One has to make sure to always pin specific versions of installed software packages. Relying mostly on the built images will make this problem less important.
	     	\end{itemize}
        \end{block}
    \end{frame}

	\section[p.five]{Conclusions and Outlook}
	{%
	\begin{frame}<beamer>
	    \frametitle{Outline}
	\tableofcontents[currentsection]
	\end{frame}
	}

	\begin{frame}
        \frametitle{Conclusions}
        \begin{block}{How to ensure reproducibility?}
            \begin{itemize}
				\item Reproducibility of scientific research will only become more important in the future.
				\pause
				\item Especially with "living data", like linguistic corpora one has to take care to ensure that findings can be reproduced by keeping older versions available.
				\pause
				\item Standard IT tools like git and docker seem to offer an easy way to handle this.
				\pause
	 			\item Still they have to be used with care.
	     	\end{itemize}
        \end{block}
    \end{frame}

	\begin{frame}
		\frametitle{The Future}
		\begin{block}{What role can CLARIN play?}
			\begin{itemize}
				\item We see two possible ways in which CLARIN can help
				\pause
				\item[1)] Develop best practices and guidelines that can help researchers in ensuring the reproducibility of their research.
				\pause
				\item[2)] Help in setting up the necessary infrastructure, for example by hosting a trusted gitlab instance that can be used to host both data and toolchain containers
			\end{itemize}
		\end{block}
	\end{frame}

	\begin{frame}
			\frametitle{Fin}
			\begin{block}{Thank you for your attention!}
			\begin{center}
					\bigskip
					\bigskip
					\textbf{Comments? Questions?}
					\\[2\baselineskip]
					\url{Alexander.Koenig@eurac.edu}
			\end{center}
			\end{block}
	\end{frame}



%%%
% REFERENCES
%
%\part{References}\frame{\partpage}
%
%	\section*{References}
%    \nocite{*}
%	\begin{frame}[allowframebreaks]
%		\frametitle{References}
%
%        % with \usepackage[...'style=authoryear'...]{biblatex}
%		%\setbeamertemplate{bibliography item}[text]
%        %\printbibliography
%
%        % without 'style=authoryear' and icons for different bibentry types
%		\setbeamertemplate{bibliography item}[online]
%		\printbibliography[ % title={Web Sites}
%            category={online}]
%		\setbeamertemplate{bibliography item}[article]
%
%		\printbibliography[ % title={Journal Articles}
%			category={article}]
%
%		\setbeamertemplate{bibliography item}[book]
%		\printbibliography[ % title={Books}
%            category={book}]
%
%        \setbeamertemplate{bibliography item}{X}
%        \printbibliography[ % title={Other Stuff}
%			category={triangle}]
%	\end{frame}
%
%
%%%

\end{document}
